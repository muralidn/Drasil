\documentclass[12pt]{article}
\usepackage{fontspec}
\usepackage{fullpage}
\usepackage{hyperref}
\hypersetup{bookmarks=true,colorlinks=true,linkcolor=red,citecolor=blue,filecolor=magenta,urlcolor=cyan}
\usepackage{amsmath}
\usepackage{amssymb}
\usepackage{mathtools}
\usepackage{unicode-math}
\usepackage{tabu}
\usepackage{longtable}
\usepackage{booktabs}
\usepackage{caption}
\usepackage{graphics}
\usepackage{enumitem}
\usepackage{filecontents}
\usepackage[backend=bibtex]{biblatex}
\usepackage{url}
\setmathfont{Latin Modern Math}
\newcommand{\gt}{\ensuremath >}
\newcommand{\lt}{\ensuremath <}
\global\tabulinesep=1mm
\newlist{symbDescription}{description}{1}
\setlist[symbDescription]{noitemsep, topsep=0pt, parsep=0pt, partopsep=0pt}
\bibliography{bibfile}
\title{Software Requirements Specification for PD Controller}
\author{Naveen Muralidharan}
\begin{document}
\maketitle
\tableofcontents
\newpage
\section{Reference Material}
\label{Sec:RefMat}
This section records information for easy reference.

\subsection{Table of Units}
\label{Sec:ToU}
The unit system used throughout is SI (Système International d'Unités). In addition to the basic units, several derived units are also used. For each unit, \hyperref[Table:ToU]{Tab: ToU} lists the symbol, a description and the SI name.

\begin{longtable}{l l l}
\toprule
\textbf{Symbol} & \textbf{Description} & \textbf{SI Name}
\\
\midrule
\endhead
${\text{s}}$ & time & second
\\
\bottomrule
\caption{Table of Units}
\label{Table:ToU}
\end{longtable}
\subsection{Table of Symbols}
\label{Sec:ToS}
The symbols used in this document are summarized in \hyperref[Table:ToS]{Tab: ToS} along with their units. The symbols are listed in alphabetical order.

\begin{longtabu}{l X[l] l}
\toprule
\textbf{Symbol} & \textbf{Description} & \textbf{Units}
\\
\midrule
\endhead
$-∞$ & Negative Infinity & --
\\
${A_{\text{tol}}}$ & Absolute tolerance & --
\\
${C_{\text{s}}}$ & Control-Variable in frequency domain & --
\\
${c_{\text{t}}}$ & Control-Variable in time domain & --
\\
${D_{\text{s}}}$ & Derivative Control in frequency domain & --
\\
${E_{\text{s}}}$ & Error Signal in the frequency domain & --
\\
${e_{\text{t}}}$ & Error Signal in the time domain & --
\\
${F_{\text{s}}}$ & Laplace Transform of a function f(t) & --
\\
${f_{\text{t}}}$ & Function in the time domain & --
\\
${H_{\text{s}}}$ & Transfer function of the Power-Plant in frequency domain & --
\\
${K_{\text{p}}}$ & Proportional Gain & --
\\
$L⁻¹{F(s)}$ & Inverse Laplace Transform of F(s) & --
\\
${P_{\text{s}}}$ & Proportional Control in frequency domain & --
\\
${R_{\text{s}}}$ & Set Point in the frequency domain & --
\\
${R_{\text{tol}}}$ & Relative tolerance & --
\\
${r_{\text{t}}}$ & Set-Point & --
\\
$s$ & Complex frequency domain parameter & --
\\
$t$ & Time & ${\text{s}}$
\\
${t_{\text{sim}}}$ & Simulation time & ${\text{s}}$
\\
${t_{\text{step}}}$ & Step time & ${\text{s}}$
\\
${Y_{\text{s}}}$ & Process Variable in the frequency domain & --
\\
${y_{\text{t}}}$ & Process Variable & --
\\
$∞$ & Infinity & --
\\
\bottomrule
\caption{Table of Symbols}
\label{Table:ToS}
\end{longtabu}
\subsection{Abbreviations and Acronyms}
\label{Sec:TAbbAcc}
\begin{longtable}{l l}
\toprule
\textbf{Abbreviation} & \textbf{Full Form}
\\
\midrule
\endhead
A & Assumption
\\
DD & Data Definition
\\
GD & General Definition
\\
GS & Goal Statement
\\
IM & Instance Model
\\
PS & Physical System Description
\\
R & Requirement
\\
SRS & Software Requirements Specification
\\
TM & Theoretical Model
\\
Uncert. & Typical Uncertainty
\\
\bottomrule
\caption{Abbreviations and Acronyms}
\label{Table:TAbbAcc}
\end{longtable}
\section{Introduction}
\label{Sec:Intro}
Automatic process control with a controller (P/PI/PD/PID) is used  in a variety of applications such as thermostats, automobile  cruise-control, etc. The gains of a controller in an application  must be tuned before the controller is ready for production. Therefore a simulation of the  PD Controller  with a  First Order System is created in this project that can be  used to tune the gain constants.

The following section provides an overview of the Software Requirements Specification (SRS) for PD Controller. This section explains the purpose of this document, the scope of the requirements, the characteristics of the intended reader, and the organization of the document.

\subsection{Purpose of Document}
\label{Sec:DocPurpose}
The purpose of this document is to capture all the necessary  information including assumptions, data definitions, constraints,  models, and requirements to facilitate an unambiguous development  of the PD controller software and test procedures.

\subsection{Scope of Requirements}
\label{Sec:ReqsScope}
The scope of the requirements includes a PD Control Loop  with three subsystems namely a  PD Controller , a Summing Point , and a Power Plant . This  software is intended to aid with the manual tuning of the PD Controller.

\subsection{Characteristics of Intended Reader}
\label{Sec:ReaderChars}
Reviewers of this documentation should have an understanding of control systems (control theory and controllers) at a fourth year undergraduate level and engineering mathematics at a second year undergraduate level. The users of PD Controller can have a lower level of expertise, as explained in \hyperref[Sec:UserChars]{Section: User Characteristics}.

\subsection{Organization of Document}
\label{Sec:DocOrg}
The sections in this document are based on  \cite{smithLai2005}. The presentation follows the standard pattern of presenting goals, theories, definitions, and assumptions. For readers that would like a more bottom up approach, they can start reading the data definitions in \hyperref[Sec:IMs]{Section: Instance Models} and trace back to find any additional information they require.

The goal statements (\hyperref[Sec:GoalStmt]{Section: Goal Statements}) are refined to the theoretical models and the theoretical models (\hyperref[Sec:TMs]{Section: Theoretical Models}) to the instance models (\hyperref[Sec:IMs]{Section: Instance Models}). .

\section{General System Description}
\label{Sec:GenSysDesc}
This section provides general information about the system. It identifies the interfaces between the system and its environment, describes the user characteristics, and lists the system constraints.

\subsection{System Context}
\label{Sec:SysContext}
\hyperref[Figure:systemContextDiag]{Fig:systemContextDiag} shows the system context. The circle represents an external entity outside the software, the user in this case. The rectangle represents the software system itself PD Controller in this case. Arrows are used to show the data flow between the system and its environment.

\begin{figure}
\begin{center}
\includegraphics[width=\textwidth]{../../../datafiles/PIDController/Fig_SystemContext.png}
\caption{System Context}
\label{Figure:systemContextDiag}
\end{center}
\end{figure}
PD Controller  is self-contained. The only external interaction is   with the  user The responsibilities of the  user  and the  system  are as follows,:

\begin{itemize}
\item{User Responsibilities}
\begin{itemize}
\item{Feed inputs to the model}
\item{Review the response of the  Power Plant}
\item{Tune the controller gains.}
\end{itemize}
\item{PD Controller Responsibilities}
\begin{itemize}
\item{Calculate the outputs of the  PD Controller  and  Power Plant}
\end{itemize}
\end{itemize}
\subsection{User Characteristics}
\label{Sec:UserChars}
The end user of  PD Controller  is expected to have completed at least the equivalent of the second year of an undergraduate degree in electrical engineering.

\subsection{System Constraints}
\label{Sec:SysConstraints}
There are no system constraints.

\section{Specific System Description}
\label{Sec:SpecSystDesc}
This section first presents the problem description, which gives a high-level view of the problem to be solved. This is followed by the solution characteristics specification, which presents the assumptions, theories, and definitions that are used.

\subsection{Problem Description}
\label{Sec:ProbDesc}
A system is needed to This program intends to provide a model of a PD Controller  that can be used for the tuning of the gain constants before  the deployment of the controller.

\subsubsection{Terminology and Definitions}
\label{Sec:TermDefs}
This subsection provides a list of terms that are used in the subsequent sections and their meaning, with the purpose of reducing ambiguity and making it easier to correctly understand the requirements.

\begin{itemize}
\item{PD Control Loop: Closed loop control system with PD Controller, Summing Point and Power Plant.}
\item{PD Controller: Proportional-Derivative Controller.}
\item{Summing Point: Control block where the difference between the Set-Point and the Process Variable is computed.}
\item{Power Plant: A first order system to be controlled.}
\item{First Order System: A system whose input-output relationship is denoted by a first order differential equation.}
\item{Error Value: Input to the PID controller. Error Value is the difference between the Set Point and the Process Variable.}
\item{Simulation time: Total execution time of the PD simulation.}
\item{Process Variable: The output value from the power plant.}
\item{Set-Point: The desired value that the control system must reach. This also knows as reference variable.}
\item{Proportional Gain: Gain constant of the proportional controller.}
\item{Derivative Gain: Gain constant of the derivative controller.}
\item{Frequency Domain: The analysis of mathematical functions with respect to frequency, instead of time.}
\item{Laplace transform: An integral transform that converts a function of a real variable t (often time) to a function of a complex variable s (complex frequency).}
\item{Control Variable: The Control Variable is the output of the PD controller.}
\item{Step time: Simulation step time.}
\end{itemize}
\subsubsection{Physical System Description}
\label{Sec:PhysSyst}
The physical system of PD Controller, as shown in \hyperref[Figure:pidSysDiagram]{Fig:pidSysDiagram}, includes the following elements:

\begin{itemize}
\item[PS1:]{The Summing Point.}
\item[PS2:]{The PD Controller.}
\item[PS3:]{The Power Plant.}
\end{itemize}
\begin{figure}
\begin{center}
\includegraphics[width=0.6\textwidth]{../../../datafiles/PIDController/Fig_PDController.png}
\caption{The physical system}
\label{Figure:pidSysDiagram}
\end{center}
\end{figure}
\subsubsection{Goal Statements}
\label{Sec:GoalStmt}
Given Set-Point, Simulation time, Proportional Gain, Derivative Gain, and Step time, the goal statements are:

\begin{itemize}
\item[Process-Variable:\phantomsection\label{processVariable}]{Calculate the output of the Power Plant ( Process Variable ) over time.}
\end{itemize}
\subsection{Solution Characteristics Specification}
\label{Sec:SolCharSpec}
The instance models that govern PD Controller are presented in \hyperref[Sec:IMs]{Section: Instance Models}. The information to understand the meaning of the instance models and their derivation is also presented, so that the instance models can be verified.

\subsubsection{Assumptions}
\label{Sec:Assumps}
This section simplifies the original problem and helps in developing the theoretical models by filling in the missing information for the physical system. The assumptions refine the scope by providing more detail.

\begin{itemize}
\item[Power plant:\phantomsection\label{pwrPlant}]{The Power Plant and the sensor are coupled  a single unit with transfer function, 1/(2s + 1). (RefBy: \hyperref[DD:ddPowerPlant]{DD: ddPowerPlant} and \hyperref[likeChgIC]{LC: Second Order Power Plant}.)}
\item[Decoupled equation:\phantomsection\label{decoupled}]{The decoupled form of the PD Controller equation used in this simulation. (RefBy: \hyperref[DD:ddCtrlVar]{DD: ddCtrlVar}.)}
\item[Set-Point:\phantomsection\label{setPoint}]{The Set-Point is a constant throughout the simulation. (RefBy: \hyperref[DD:ddErrorSignal]{DD: ddErrorSignal} and \hyperref[IM:pdEquationIM]{IM: pdEquationIM}.)}
\item[External disturbance:\phantomsection\label{externalDistub}]{There are no external disturbances to the Power Plant during the simulation. (RefBy: \hyperref[DD:ddPowerPlant]{DD: ddPowerPlant}.)}
\item[Initial value:\phantomsection\label{initialValue}]{The initial value of the Process Variable is assumed to be zero. (RefBy: \hyperref[DD:ddErrorSignal]{DD: ddErrorSignal}.)}
\item[Parallel equation:\phantomsection\label{parallelEq}]{The Parallel form of the equation is used for the PD Controller. (RefBy: \hyperref[DD:ddCtrlVar]{DD: ddCtrlVar}.)}
\end{itemize}
\subsubsection{Theoretical Models}
\label{Sec:TMs}
This section focuses on the general equations and laws that PD Controller is based on.

\vspace{\baselineskip}
\noindent
\begin{minipage}{\textwidth}
\begin{tabular}{>{\raggedright}p{0.13\textwidth}>{\raggedright\arraybackslash}p{0.82\textwidth}}
\toprule \textbf{Refname} & \textbf{TM:laplaceTransform}
\phantomsection 
\label{TM:laplaceTransform}
\\ \midrule \\
Label & Laplace transform
        
\\ \midrule \\
Equation & \begin{displaymath}
           {F_{\text{s}}}=\int_{-∞}^{∞}{{f_{\text{t}}} e^{-s t}}\,dt
           \end{displaymath}
\\ \midrule \\
Description & \begin{symbDescription}
              \item{${F_{\text{s}}}$ is the Laplace Transform of a function f(t) (Unitless)}
              \item{${f_{\text{t}}}$ is the Function in the time domain (Unitless)}
              \item{$s$ is the Complex frequency domain parameter (Unitless)}
              \item{$t$ is the time (${\text{s}}$)}
              \end{symbDescription}
\\ \midrule \\
Notes & Bilateral Laplace Transform. The Laplace transforms are  usually inferred from the Laplace Transform table in   section 4 of the  \cite{laplaceWiki}.
        
\\ \midrule \\
Source & \cite{laplaceWiki}
         
\\ \midrule \\
RefBy & \hyperref[DD:ddPowerPlant]{DD: ddPowerPlant}, \hyperref[DD:ddPropCtrl]{DD: ddPropCtrl}, \hyperref[DD:ddErrorSignal]{DD: ddErrorSignal}, and \hyperref[DD:ddDerivCtrl]{DD: ddDerivCtrl}
        
\\ \bottomrule
\end{tabular}
\end{minipage}
\vspace{\baselineskip}
\noindent
\begin{minipage}{\textwidth}
\begin{tabular}{>{\raggedright}p{0.13\textwidth}>{\raggedright\arraybackslash}p{0.82\textwidth}}
\toprule \textbf{Refname} & \textbf{TM:invLaplaceTransform}
\phantomsection 
\label{TM:invLaplaceTransform}
\\ \midrule \\
Label & Inverse laplace transform
        
\\ \midrule \\
Equation & \begin{displaymath}
           {f_{\text{t}}}=L⁻¹{F(s)}
           \end{displaymath}
\\ \midrule \\
Description & \begin{symbDescription}
              \item{${f_{\text{t}}}$ is the Function in the time domain (Unitless)}
              \item{$L⁻¹{F(s)}$ is the Inverse Laplace Transform of F(s) (Unitless)}
              \end{symbDescription}
\\ \midrule \\
Notes & Inverse Laplace Transform of F(S). The Inverse Laplace transforms are  usually inferred from the Laplace Transform table in   section 4 of the  \cite{laplaceWiki}.
        
\\ \midrule \\
Source & \cite{laplaceWiki}
         
\\ \midrule \\
RefBy & \hyperref[IM:pdEquationIM]{IM: pdEquationIM}
        
\\ \bottomrule
\end{tabular}
\end{minipage}
\subsubsection{General Definitions}
\label{Sec:GDs}
There are no general definitions.

\subsubsection{Data Definitions}
\label{Sec:DDs}
This section collects and defines all the data needed to build the instance models.

\vspace{\baselineskip}
\noindent
\begin{minipage}{\textwidth}
\begin{tabular}{>{\raggedright}p{0.13\textwidth}>{\raggedright\arraybackslash}p{0.82\textwidth}}
\toprule \textbf{Refname} & \textbf{DD:ddErrorSignal}
\phantomsection 
\label{DD:ddErrorSignal}
\\ \midrule \\
Label & Error Signal in the frequency domain
        
\\ \midrule \\
Symbol & ${E_{\text{s}}}$
         
\\ \midrule \\
Units & Unitless
        
\\ \midrule \\
Equation & \begin{displaymath}
           {E_{\text{s}}}={R_{\text{s}}}-{Y_{\text{s}}}
           \end{displaymath}
\\ \midrule \\
Description & \begin{symbDescription}
              \item{${E_{\text{s}}}$ is the Error Signal in the frequency domain (Unitless)}
              \item{${R_{\text{s}}}$ is the Set Point in the frequency domain (Unitless)}
              \item{${Y_{\text{s}}}$ is the Process Variable in the frequency domain (Unitless)}
              \end{symbDescription}
\\ \midrule \\
Notes & Error Signal is the difference between the Set-Point and  Process Variable. The equation is converted to frequency domain by applying the Laplace transform ( from \hyperref[TM:laplaceTransform]{TM: laplaceTransform} ). The Set-Point is assumed to be constant throughout the simulation ( from  \hyperref[setPoint]{A: Set-Point} ). The initial value of the Process Variable if assumed to be zero ( from  \hyperref[initialValue]{A: Initial value} )..
        
\\ \midrule \\
Source & \cite{johnson2008}
         
\\ \midrule \\
RefBy & \hyperref[DD:ddPropCtrl]{DD: ddPropCtrl}, \hyperref[DD:ddDerivCtrl]{DD: ddDerivCtrl}, and \hyperref[IM:pdEquationIM]{IM: pdEquationIM}
        
\\ \bottomrule
\end{tabular}
\end{minipage}

\vspace{\baselineskip}
\noindent
\begin{minipage}{\textwidth}
\begin{tabular}{>{\raggedright}p{0.13\textwidth}>{\raggedright\arraybackslash}p{0.82\textwidth}}
\toprule \textbf{Refname} & \textbf{DD:ddPropCtrl}
\phantomsection 
\label{DD:ddPropCtrl}
\\ \midrule \\
Label & Proportional Control in frequency domain
        
\\ \midrule \\
Symbol & ${P_{\text{s}}}$
         
\\ \midrule \\
Units & Unitless
        
\\ \midrule \\
Equation & \begin{displaymath}
           {P_{\text{s}}}={K_{\text{d}}}\cdot{}{E_{\text{s}}}
           \end{displaymath}
\\ \midrule \\
Description & \begin{symbDescription}
              \item{${P_{\text{s}}}$ is the Proportional Control in frequency domain (Unitless)}
              \item{${K_{\text{d}}}$ is the Proportional Gain (Unitless)}
              \item{${E_{\text{s}}}$ is the Error Signal in the frequency domain (Unitless)}
              \end{symbDescription}
\\ \midrule \\
Notes & Proportional controller is the product of the Proportional Gain and the Error Signal ( from  \hyperref[DD:ddErrorSignal]{DD: ddErrorSignal} ) The equation is converted to frequency domain by applying the Laplace transform ( from \hyperref[TM:laplaceTransform]{TM: laplaceTransform} ).
        
\\ \midrule \\
Source & \cite{johnson2008}
         
\\ \midrule \\
RefBy & \hyperref[DD:ddCtrlVar]{DD: ddCtrlVar}
        
\\ \bottomrule
\end{tabular}
\end{minipage}

\vspace{\baselineskip}
\noindent
\begin{minipage}{\textwidth}
\begin{tabular}{>{\raggedright}p{0.13\textwidth}>{\raggedright\arraybackslash}p{0.82\textwidth}}
\toprule \textbf{Refname} & \textbf{DD:ddDerivCtrl}
\phantomsection 
\label{DD:ddDerivCtrl}
\\ \midrule \\
Label & Derivative Control in frequency domain
        
\\ \midrule \\
Symbol & ${D_{\text{s}}}$
         
\\ \midrule \\
Units & Unitless
        
\\ \midrule \\
Equation & \begin{displaymath}
           {D_{\text{s}}}={K_{\text{d}}}\cdot{}{E_{\text{s}}}\cdot{}s
           \end{displaymath}
\\ \midrule \\
Description & \begin{symbDescription}
              \item{${D_{\text{s}}}$ is the Derivative Control in frequency domain (Unitless)}
              \item{${K_{\text{d}}}$ is the Proportional Gain (Unitless)}
              \item{${E_{\text{s}}}$ is the Error Signal in the frequency domain (Unitless)}
              \item{$s$ is the Complex frequency domain parameter (Unitless)}
              \end{symbDescription}
\\ \midrule \\
Notes & Derivative controller is the product of the Derivative Gain and the differential of the Error Signal ( from  \hyperref[DD:ddErrorSignal]{DD: ddErrorSignal} ) The equation is converted to frequency domain by applying the Laplace transform ( from \hyperref[TM:laplaceTransform]{TM: laplaceTransform} ).
        
\\ \midrule \\
Source & \cite{johnson2008}
         
\\ \midrule \\
RefBy & \hyperref[DD:ddCtrlVar]{DD: ddCtrlVar}
        
\\ \bottomrule
\end{tabular}
\end{minipage}

\vspace{\baselineskip}
\noindent
\begin{minipage}{\textwidth}
\begin{tabular}{>{\raggedright}p{0.13\textwidth}>{\raggedright\arraybackslash}p{0.82\textwidth}}
\toprule \textbf{Refname} & \textbf{DD:ddPowerPlant}
\phantomsection 
\label{DD:ddPowerPlant}
\\ \midrule \\
Label & Transfer function of the Power-Plant in frequency domain
        
\\ \midrule \\
Symbol & ${H_{\text{s}}}$
         
\\ \midrule \\
Units & Unitless
        
\\ \midrule \\
Equation & \begin{displaymath}
           {H_{\text{s}}}=\frac{1}{2 s+1}
           \end{displaymath}
\\ \midrule \\
Description & \begin{symbDescription}
              \item{${H_{\text{s}}}$ is the Transfer function of the Power-Plant in frequency domain (Unitless)}
              \item{$s$ is the Complex frequency domain parameter (Unitless)}
              \end{symbDescription}
\\ \midrule \\
Notes & The power plant is represented by a first order system ( from  \hyperref[pwrPlant]{A: Power plant} ) The equation is converted to frequency domain by applying the Laplace transform ( from \hyperref[TM:laplaceTransform]{TM: laplaceTransform} ) Additionally there are no external disturbances to the power plant ( from  \hyperref[externalDistub]{A: External disturbance} ).
        
\\ \midrule \\
Source & \cite{pidWiki}
         
\\ \midrule \\
RefBy & \hyperref[IM:pdEquationIM]{IM: pdEquationIM}
        
\\ \bottomrule
\end{tabular}
\end{minipage}

\vspace{\baselineskip}
\noindent
\begin{minipage}{\textwidth}
\begin{tabular}{>{\raggedright}p{0.13\textwidth}>{\raggedright\arraybackslash}p{0.82\textwidth}}
\toprule \textbf{Refname} & \textbf{DD:ddCtrlVar}
\phantomsection 
\label{DD:ddCtrlVar}
\\ \midrule \\
Label & Control-Variable in frequency domain
        
\\ \midrule \\
Symbol & ${C_{\text{s}}}$
         
\\ \midrule \\
Units & Unitless
        
\\ \midrule \\
Equation & \begin{displaymath}
           {C_{\text{s}}}={K_{\text{d}}}\cdot{}{E_{\text{s}}}+{K_{\text{d}}}\cdot{}{E_{\text{s}}}\cdot{}s
           \end{displaymath}
\\ \midrule \\
Description & \begin{symbDescription}
              \item{${C_{\text{s}}}$ is the Control-Variable in frequency domain (Unitless)}
              \item{${K_{\text{d}}}$ is the Proportional Gain (Unitless)}
              \item{${E_{\text{s}}}$ is the Error Signal in the frequency domain (Unitless)}
              \item{$s$ is the Complex frequency domain parameter (Unitless)}
              \end{symbDescription}
\\ \midrule \\
Notes & The control variable is the output of the controller. In this case, it is the sum of the Proportional ( from \hyperref[DD:ddPropCtrl]{DD: ddPropCtrl} ) and Derivative ( from  \hyperref[DD:ddDerivCtrl]{DD: ddDerivCtrl} ) controllers The parallel ( from \hyperref[parallelEq]{A: Parallel equation} ) and de-coupled ( from \hyperref[decoupled]{A: Decoupled equation} ) form of the PD equation is used in this document.
        
\\ \midrule \\
Source & \cite{johnson2008}
         
\\ \midrule \\
RefBy & \hyperref[IM:pdEquationIM]{IM: pdEquationIM}
        
\\ \bottomrule
\end{tabular}
\end{minipage}

\subsubsection{Instance Models}
\label{Sec:IMs}
This section transforms the problem defined in \hyperref[Sec:ProbDesc]{Section: Problem Description} into one which is expressed in mathematical terms. It uses concrete symbols defined in \hyperref[Sec:DDs]{Section: Data Definitions} to replace the abstract symbols in the models identified in \hyperref[Sec:TMs]{Section: Theoretical Models} and \hyperref[Sec:GDs]{Section: General Definitions}.

\vspace{\baselineskip}
\noindent
\begin{minipage}{\textwidth}
\begin{tabular}{>{\raggedright}p{0.13\textwidth}>{\raggedright\arraybackslash}p{0.82\textwidth}}
\toprule \textbf{Refname} & \textbf{IM:pdEquationIM}
\phantomsection 
\label{IM:pdEquationIM}
\\ \midrule \\
Label & Computation of the Process Variable as a function of time.
        
\\ \midrule \\
Input & ${r_{\text{t}}}$, ${K_{\text{p}}}$, ${K_{\text{d}}}$
        
\\ \midrule \\
Output & ${y_{\text{t}}}$
         
\\ \midrule \\
Input Constraints & \begin{displaymath}
                    {r_{\text{t}}}\gt{}0
                    \end{displaymath}
                    \begin{displaymath}
                    {K_{\text{d}}}\gt{}0
                    \end{displaymath}
                    \begin{displaymath}
                    {K_{\text{d}}}\gt{}0
                    \end{displaymath}
\\ \midrule \\
Output Constraints & \begin{displaymath}
                     {y_{\text{t}}}\gt{}0
                     \end{displaymath}
\\ \midrule \\
Equation & \begin{displaymath}
           \left(2+{K_{\text{d}}}\right) \frac{\,d{y_{\text{t}}}}{\,dt}+\left(1+{K_{\text{d}}}\right) {y_{\text{t}}}-{r_{\text{t}}} {K_{\text{d}}}=0
           \end{displaymath}
\\ \midrule \\
Description & \begin{symbDescription}
              \item{${K_{\text{d}}}$ is the Proportional Gain (Unitless)}
              \item{$t$ is the time (${\text{s}}$)}
              \item{${y_{\text{t}}}$ is the Process Variable (Unitless)}
              \item{${r_{\text{t}}}$ is the Set-Point (Unitless)}
              \end{symbDescription}
\\ \midrule \\
Source & \cite{abbasi2015} and \cite{johnson2008}
         
\\ \midrule \\
RefBy & \hyperref[outputValues]{FR: Output-Values} and \hyperref[calculateValues]{FR: Calculate-Values}
        
\\ \bottomrule
\end{tabular}
\end{minipage}
\paragraph{Detailed derivation of Process Variable:}
\label{IM:pdEquationIMDeriv}
The Process Variable (Y(S)) in a PD Control Loop is the product of the Error signal ( from  \hyperref[DD:ddErrorSignal]{DD: ddErrorSignal} ), Control Variable ( from \hyperref[DD:ddCtrlVar]{DD: ddCtrlVar}  ), and the Power-Plant ( from \hyperref[DD:ddPowerPlant]{DD: ddPowerPlant} ).

\begin{displaymath}
{Y_{\text{s}}}=\left({R_{\text{s}}}-{Y_{\text{s}}}\right) \left({K_{\text{d}}}+{K_{\text{d}}} s\right) \frac{1}{2 s+1}
\end{displaymath}
Rearranging the equation.

\begin{displaymath}
\left(2+{K_{\text{d}}}\right) {Y_{\text{s}}} s+\left(1+{K_{\text{d}}}\right) {Y_{\text{s}}}-{R_{\text{s}}} s {K_{\text{d}}}-{R_{\text{s}}} {K_{\text{d}}}=0
\end{displaymath}
Computing the Inverse Laplace Transform of F(s) (from  \hyperref[TM:invLaplaceTransform]{TM: invLaplaceTransform} ) of the equation.

\begin{displaymath}
\left(2+{K_{\text{d}}}\right) \frac{\,d{y_{\text{t}}}}{\,dt}+\left(1+{K_{\text{d}}}\right) {y_{\text{t}}}-{K_{\text{d}}} \frac{\,d{r_{\text{t}}}}{\,dt}-{r_{\text{t}}} {K_{\text{d}}}=0
\end{displaymath}
The Set point (r(t)) is a step function, and a constant  ( from  \hyperref[setPoint]{A: Set-Point} ). Therefore the  differential of the set point is zero. Hence the equation  reduces to,

\begin{displaymath}
\left(2+{K_{\text{d}}}\right) \frac{\,d{y_{\text{t}}}}{\,dt}+\left(1+{K_{\text{d}}}\right) {y_{\text{t}}}-{r_{\text{t}}} {K_{\text{d}}}=0
\end{displaymath}
\subsubsection{Data Constraints}
\label{Sec:DataConstraints}
\hyperref[Table:InDataConstraints]{Table:InDataConstraints} shows the data constraints on the input variables. The column for physical constraints gives the physical limitations on the range of values that can be taken by the variable. The uncertainty column provides an estimate of the confidence with which the physical quantities can be measured. This information would be part of the input if one were performing an uncertainty quantification exercise. The constraints are conservative, to give the user of the model the flexibility to experiment with unusual situations. The column of typical values is intended to provide a feel for a common scenario.

\begin{longtable}{l l l l}
\toprule
\textbf{Var} & \textbf{Physical Constraints} & \textbf{Typical Value} & \textbf{Uncert.}
\\
\midrule
\endhead
${K_{\text{d}}}$ & ${K_{\text{d}}}\gt{}0$ & $1.0$ & 10$\%$
\\
${K_{\text{d}}}$ & ${K_{\text{d}}}\gt{}0$ & $100.0$ & 10$\%$
\\
${r_{\text{t}}}$ & ${r_{\text{t}}}\gt{}0$ & $1.0$ & 10$\%$
\\
${t_{\text{sim}}}$ & $1\leq{}{t_{\text{sim}}}\leq{}60$ & $10.0$ ${\text{s}}$ & 10$\%$
\\
${t_{\text{step}}}$ & $\frac{1}{100}\leq{}{t_{\text{step}}}\leq{}1$ & $0.01$ ${\text{s}}$ & 10$\%$
\\
\bottomrule
\caption{Input Data Constraints}
\label{Table:InDataConstraints}
\end{longtable}
\section{Requirements}
\label{Sec:Requirements}
This section provides the functional requirements, the tasks and behaviours that the software is expected to complete, and the non-functional requirements, the qualities that the software is expected to exhibit.

\subsection{Functional Requirements}
\label{Sec:FRs}
This section provides the functional requirements, the tasks and behaviours that the software is expected to complete.

\begin{itemize}
\item[Input-Values:\phantomsection\label{inputValues}]{Input the values from \hyperref[Table:ReqInputs]{Table:ReqInputs}.}
\item[Verify-Input-Values:\phantomsection\label{verifyInputs}]{Ensure that the input values are within the limits specified in \hyperref[Sec:DataConstraints]{Section: Data Constraints}..}
\item[Calculate-Values:\phantomsection\label{calculateValues}]{Calculate the Process Variable ( from \hyperref[IM:pdEquationIM]{IM: pdEquationIM}  ) over  the simulation time.}
\item[Output-Values:\phantomsection\label{outputValues}]{Output the Process Variable ( from \hyperref[IM:pdEquationIM]{IM: pdEquationIM}  ) over  the simulation time.}
\end{itemize}
\begin{longtable}{l l l}
\toprule
\textbf{Symbol} & \textbf{Description} & \textbf{Units}
\\
\midrule
\endhead
${A_{\text{tol}}}$ & Absolute tolerance & --
\\
${K_{\text{d}}}$ & Proportional Gain & --
\\
${K_{\text{d}}}$ & Proportional Gain & --
\\
${R_{\text{tol}}}$ & Relative tolerance & --
\\
${r_{\text{t}}}$ & Set-Point & --
\\
${t_{\text{sim}}}$ & Simulation time & ${\text{s}}$
\\
${t_{\text{step}}}$ & Step time & ${\text{s}}$
\\
\bottomrule
\caption{Required Inputs following \hyperref[inputValues]{FR: Input-Values}}
\label{Table:ReqInputs}
\end{longtable}
\subsection{Non-Functional Requirements}
\label{Sec:NFRs}
This section provides the non-functional requirements, the qualities that the software is expected to exhibit.

\begin{itemize}
\item[Portable:\phantomsection\label{portability}]{The code shall be portable to multiple Operating Systems.}
\item[Secure:\phantomsection\label{security}]{The code shall be immune to common security problems such as memory leaks, divide by zero errors, and the square root of negative numbers.}
\item[Maintainable:\phantomsection\label{maintainability}]{The code shall be thoroughly documented with appropriate User Guides.}
\item[Verifiable:\phantomsection\label{verifiability}]{The code shall be verifiable against a Verification and Validation plan.}
\item[Quality:\phantomsection\label{quality}]{The code shall be written with high-quality standards. The code should adhere to good coding standards and should not contain any dead, or unreachable statements..}
\end{itemize}
\section{Likely Changes}
\label{Sec:LCs}
This section lists the likely changes to be made to the software.

\begin{itemize}
\item[Second Order Power Plant:\phantomsection\label{likeChgIC}]{The Power Plant maybe changed into a second order system ( from  \hyperref[pwrPlant]{A: Power plant} ).}
\end{itemize}
\section{Traceability Matrices and Graphs}
\label{Sec:TraceMatrices}
The purpose of the traceability matrices is to provide easy references on what has to be additionally modified if a certain component is changed. Every time a component is changed, the items in the column of that component that are marked with an ``X'' should be modified as well. \hyperref[Table:TraceMatAvsA]{Table:TraceMatAvsA} shows the dependencies of assumptions on the assumptions. \hyperref[Table:TraceMatAvsAll]{Table:TraceMatAvsAll} shows the dependencies of data definitions, theoretical models, general definitions, instance models, requirements, likely changes, and unlikely changes on the assumptions. \hyperref[Table:TraceMatRefvsRef]{Table:TraceMatRefvsRef} shows the dependencies of data definitions, theoretical models, general definitions, and instance models with each other. \hyperref[Table:TraceMatAllvsR]{Table:TraceMatAllvsR} shows the dependencies of requirements, goal statements on the data definitions, theoretical models, general definitions, and instance models.

\begin{longtable}{l l l l l l l}
\toprule
\textbf{} & \textbf{\hyperref[pwrPlant]{A: Power plant}} & \textbf{\hyperref[decoupled]{A: Decoupled equation}} & \textbf{\hyperref[setPoint]{A: Set-Point}} & \textbf{\hyperref[externalDistub]{A: External disturbance}} & \textbf{\hyperref[initialValue]{A: Initial value}} & \textbf{\hyperref[parallelEq]{A: Parallel equation}}
\\
\midrule
\endhead
\hyperref[pwrPlant]{A: Power plant} &  &  &  &  &  & 
\\
\hyperref[decoupled]{A: Decoupled equation} &  &  &  &  &  & 
\\
\hyperref[setPoint]{A: Set-Point} &  &  &  &  &  & 
\\
\hyperref[externalDistub]{A: External disturbance} &  &  &  &  &  & 
\\
\hyperref[initialValue]{A: Initial value} &  &  &  &  &  & 
\\
\hyperref[parallelEq]{A: Parallel equation} &  &  &  &  &  & 
\\
\bottomrule
\caption{Traceability Matrix Showing the Connections Between Assumptions dependence of each other.}
\label{Table:TraceMatAvsA}
\end{longtable}
\begin{longtable}{l l l l l l l}
\toprule
\textbf{} & \textbf{\hyperref[pwrPlant]{A: Power plant}} & \textbf{\hyperref[decoupled]{A: Decoupled equation}} & \textbf{\hyperref[setPoint]{A: Set-Point}} & \textbf{\hyperref[externalDistub]{A: External disturbance}} & \textbf{\hyperref[initialValue]{A: Initial value}} & \textbf{\hyperref[parallelEq]{A: Parallel equation}}
\\
\midrule
\endhead
\hyperref[DD:ddErrorSignal]{DD: ddErrorSignal} &  &  & X &  & X & 
\\
\hyperref[DD:ddPropCtrl]{DD: ddPropCtrl} &  &  &  &  &  & 
\\
\hyperref[DD:ddDerivCtrl]{DD: ddDerivCtrl} &  &  &  &  &  & 
\\
\hyperref[DD:ddPowerPlant]{DD: ddPowerPlant} & X &  &  & X &  & 
\\
\hyperref[DD:ddCtrlVar]{DD: ddCtrlVar} &  & X &  &  &  & X
\\
\hyperref[TM:laplaceTransform]{TM: laplaceTransform} &  &  &  &  &  & 
\\
\hyperref[TM:invLaplaceTransform]{TM: invLaplaceTransform} &  &  &  &  &  & 
\\
\hyperref[IM:pdEquationIM]{IM: pdEquationIM} &  &  & X &  &  & 
\\
\hyperref[inputValues]{FR: Input-Values} &  &  &  &  &  & 
\\
\hyperref[verifyInputs]{FR: Verify-Input-Values} &  &  &  &  &  & 
\\
\hyperref[calculateValues]{FR: Calculate-Values} &  &  &  &  &  & 
\\
\hyperref[outputValues]{FR: Output-Values} &  &  &  &  &  & 
\\
\hyperref[portability]{NFR: Portable} &  &  &  &  &  & 
\\
\hyperref[security]{NFR: Secure} &  &  &  &  &  & 
\\
\hyperref[maintainability]{NFR: Maintainable} &  &  &  &  &  & 
\\
\hyperref[verifiability]{NFR: Verifiable} &  &  &  &  &  & 
\\
\hyperref[quality]{NFR: Quality} &  &  &  &  &  & 
\\
\hyperref[likeChgIC]{LC: Second Order Power Plant} & X &  &  &  &  & 
\\
\bottomrule
\caption{Traceability Matrix Showing the Connections Between Assumptions and Other Items}
\label{Table:TraceMatAvsAll}
\end{longtable}
\begin{longtable}{l l l l l l l l l}
\toprule
\textbf{} & \textbf{\hyperref[DD:ddErrorSignal]{DD: ddErrorSignal}} & \textbf{\hyperref[DD:ddPropCtrl]{DD: ddPropCtrl}} & \textbf{\hyperref[DD:ddDerivCtrl]{DD: ddDerivCtrl}} & \textbf{\hyperref[DD:ddPowerPlant]{DD: ddPowerPlant}} & \textbf{\hyperref[DD:ddCtrlVar]{DD: ddCtrlVar}} & \textbf{\hyperref[TM:laplaceTransform]{TM: laplaceTransform}} & \textbf{\hyperref[TM:invLaplaceTransform]{TM: invLaplaceTransform}} & \textbf{\hyperref[IM:pdEquationIM]{IM: pdEquationIM}}
\\
\midrule
\endhead
\hyperref[DD:ddErrorSignal]{DD: ddErrorSignal} &  &  &  &  &  & X &  & 
\\
\hyperref[DD:ddPropCtrl]{DD: ddPropCtrl} & X &  &  &  &  & X &  & 
\\
\hyperref[DD:ddDerivCtrl]{DD: ddDerivCtrl} & X &  &  &  &  & X &  & 
\\
\hyperref[DD:ddPowerPlant]{DD: ddPowerPlant} &  &  &  &  &  & X &  & 
\\
\hyperref[DD:ddCtrlVar]{DD: ddCtrlVar} &  & X & X &  &  &  &  & 
\\
\hyperref[TM:laplaceTransform]{TM: laplaceTransform} &  &  &  &  &  &  &  & 
\\
\hyperref[TM:invLaplaceTransform]{TM: invLaplaceTransform} &  &  &  &  &  &  &  & 
\\
\hyperref[IM:pdEquationIM]{IM: pdEquationIM} & X &  &  & X & X &  & X & 
\\
\bottomrule
\caption{Traceability Matrix Showing the Connections Between Items and Other Sections}
\label{Table:TraceMatRefvsRef}
\end{longtable}
\begin{longtable}{l l l l l l l l l l l l l l l l l l}
\toprule
\textbf{} & \textbf{\hyperref[DD:ddErrorSignal]{DD: ddErrorSignal}} & \textbf{\hyperref[DD:ddPropCtrl]{DD: ddPropCtrl}} & \textbf{\hyperref[DD:ddDerivCtrl]{DD: ddDerivCtrl}} & \textbf{\hyperref[DD:ddPowerPlant]{DD: ddPowerPlant}} & \textbf{\hyperref[DD:ddCtrlVar]{DD: ddCtrlVar}} & \textbf{\hyperref[TM:laplaceTransform]{TM: laplaceTransform}} & \textbf{\hyperref[TM:invLaplaceTransform]{TM: invLaplaceTransform}} & \textbf{\hyperref[IM:pdEquationIM]{IM: pdEquationIM}} & \textbf{\hyperref[inputValues]{FR: Input-Values}} & \textbf{\hyperref[verifyInputs]{FR: Verify-Input-Values}} & \textbf{\hyperref[calculateValues]{FR: Calculate-Values}} & \textbf{\hyperref[outputValues]{FR: Output-Values}} & \textbf{\hyperref[portability]{NFR: Portable}} & \textbf{\hyperref[security]{NFR: Secure}} & \textbf{\hyperref[maintainability]{NFR: Maintainable}} & \textbf{\hyperref[verifiability]{NFR: Verifiable}} & \textbf{\hyperref[quality]{NFR: Quality}}
\\
\midrule
\endhead
\hyperref[processVariable]{GS: Process-Variable} &  &  &  &  &  &  &  &  &  &  &  &  &  &  &  &  & 
\\
\hyperref[inputValues]{FR: Input-Values} &  &  &  &  &  &  &  &  &  &  &  &  &  &  &  &  & 
\\
\hyperref[verifyInputs]{FR: Verify-Input-Values} &  &  &  &  &  &  &  &  &  &  &  &  &  &  &  &  & 
\\
\hyperref[calculateValues]{FR: Calculate-Values} &  &  &  &  &  &  &  & X &  &  &  &  &  &  &  &  & 
\\
\hyperref[outputValues]{FR: Output-Values} &  &  &  &  &  &  &  & X &  &  &  &  &  &  &  &  & 
\\
\hyperref[portability]{NFR: Portable} &  &  &  &  &  &  &  &  &  &  &  &  &  &  &  &  & 
\\
\hyperref[security]{NFR: Secure} &  &  &  &  &  &  &  &  &  &  &  &  &  &  &  &  & 
\\
\hyperref[maintainability]{NFR: Maintainable} &  &  &  &  &  &  &  &  &  &  &  &  &  &  &  &  & 
\\
\hyperref[verifiability]{NFR: Verifiable} &  &  &  &  &  &  &  &  &  &  &  &  &  &  &  &  & 
\\
\hyperref[quality]{NFR: Quality} &  &  &  &  &  &  &  &  &  &  &  &  &  &  &  &  & 
\\
\bottomrule
\caption{Traceability Matrix Showing the Connections Between Requirements, Goal Statements and Other Items}
\label{Table:TraceMatAllvsR}
\end{longtable}
\section{References}
\label{Sec:References}
\begin{filecontents*}{bibfile.bib}
@book{johnson2008,
author={Johnson, Michael A. and Moradi, Mohammad H.},
title={PID Control: New Identification and Design Methods, Chapter 1},
publisher={Springer Science and Business Media},
year={2006}}
@misc{abbasi2015,
author={Nasser M. Abbasi},
title={A differential equation view of closed loop control systems},
howpublished={\url{https://www.12000.org/my\_notes/connecting\_systems/report.htm}},
month=nov,
year={2020}}
@misc{laplaceWiki,
author={Wikipedia Contributors},
title={Laplace transform},
howpublished={\url{https://en.wikipedia.org/wiki/Laplace\_transform}},
month=nov,
year={2020}}
@misc{pidWiki,
author={Wikipedia Contributors},
title={PID controller},
howpublished={\url{https://en.wikipedia.org/wiki/PID\_controller}},
month=oct,
year={2020}}
\end{filecontents*}
\nocite{*}
\bibstyle{ieeetr}
\printbibliography[heading=none]
\end{document}
